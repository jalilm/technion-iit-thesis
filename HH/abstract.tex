\section{abstract}
The detection of of Heavy Hitter (HH) flows in a network device is a critical building block in many control and management tasks.  A flow is considered a Heavy Hitter flow if its portion from the total traffic surpasses a given threshold.  One of the most important aspect of this detection is its practicality; i.e., being able to work in line rate using the available scarce local memory in the device.  

In this paper, we present a practical heavy hitters detection algorithm that requires a constant amount of memory (not related to the number of flows or the number of packets) and performs at most $O(1)$ operation per packet to keep with line rate speed. We present an analysis of errors for our algorithm and compare it to state-of-the-art monitoring solutions, showing a superior performance where the allocated memory is less than $1MB$. In particular, we are able to detect more HH flows with less false positive without increasing the per-packet processing time.