\section{Analysis of Errors}
\label{sec:theory}
In this section we study the types of errors introduced by our algorithm and provide an analysis for each of them. In order to do so we denote by $F_{h_1(f)}$ the set of flows that are mapped to the same entry by $h_1$ as $f$, i.e., $F_{h_1(f)} = \{g\in F | h_1(g)=h_1(f)\}$, and by $\gamma_f$ the portion of flow $f$ from the whole trace, i.e., $\gamma_f N$ is the number of packets of flow $f$. Furthermore, we denote by $\Gamma_f$ the sum of portions of flows mapped to the same entry as $f$ by $h_1$, i.e., $\Gamma_f=\sum_{g\in F_{h_1(f)}} \gamma_g$.

The total error in the frequency estimation of a HH flow might originate from the various parts of the algorithm. The first type of error type is the \textit{\ee} which originates from the usage of shared estimators in the \sea. This relative error is bounded by $\delta$ as guaranteed from the original CEDAR scheme.

The second error type, which we denote by \textit{\pe}, is the error introduced to the estimation of HH flows for not propagating early enough from the \sfa\ to the \cs. While indeed, the algorithm will give a frequency estimation within the relative error of the true frequency for any flow in the \cs, however, this estimation holds from the moment the flow was propagated to the \cs, and this propagation does not happen on the first packet of the flow. The \pe\ captures the number of times the flow appeared in the trace before propagating to \cs.

When examining the $i^{th}$ and the ${i+1}^{th}$ packets of a HH flow $f$ (for $i\geq 2$), the probability of $f$ being evicted from the \sfa\ before propagating to the \cs\ is $\frac{\Gamma_f-\gamma_f}{\Gamma_f}$. This is true, since $\Gamma_F-\gamma_f$ is the portion of the flows but $f$ that are mapped to the same entry as $f$. Furthermore, the probability of a HH flow $f$ not to propagate to the \cs\ after $i+1$ packets is $\left(\frac{\Gamma_f-\gamma_f}{\gamma_g}\right)^{i}$, which approaches $0$ as more packets for flow $f$ arrive.

The last type of error, which we denote by \textit{\eve}, is the error introduced to the estimation of HH flows due to the eviction of such flows from the \cs. These evictions happen when there is a collision in $h_2$, then the algorithm evicts the currently tracked flow with a low probability of $\frac{v_0}{v_i}$. The motivation of evicting with this probability is to replace the current flow, only if the new flow had arrived many times. However, evictions of HH flows could still occur due to the possible collisions in $h_2$ with other HH flows.

Since $h_2$ is a uniform hash function, the expected number of collisions between HH flows is given by $\frac{h(h-1)}{2M}$ where $h$ is the number of HH flows and $M$ is the number of entries in the \cs. It is clear that the more memory used for the \cs\ the less expected collisions we will have in $h_2$ and the smaller the \eve.
