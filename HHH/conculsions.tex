\section{Conclusions}
In this paper we presented several practical algorithms for  Hierarchical Heavy Hitters detection.  These algorithms can be deployed on off-the-shelf network nodes (or software devices) and can operate in line speed due to their $O(1)$ per-packet operation. The current state of the art algorithms, either require $O(H)$ per-packet operation that makes them unfeasible to be deployed in line rate or requires a convergence interval before reporting satisfactory results which makes them less relevant in many practical settings. In contrary, our algorithms perform in line-speed with $O(1)$ update per-packet without requiring any convergence interval. Furthermore, no complex data structures are needed and our algorithms only require using built-in fast counters available in any network node.

We evaluated the algorithms on two recent real Internet packets traces and showed that they yield a comparable results to the state of the art without their limitations. The evaluation showed that the best algorithm can detect up to 90\% of the HHH in a trace and report no more than 5\% non HHH flows.

These algorithms could be easily extend to the case of multi-dimensional HHH while keeping the depth of the hierarchy linear in the number of dimensions without modifying the $O(1)$ update time. Also, they allow practical detection of the weighted set of HHH flows with minimal modification of the update operations while keeping all of the algorithms promises.
In future work, we plan to study the control mechanism of the algorithms and their fine deployment issues. Furthermore, we plan to adjust the algorithms to detect DDoS attacks by facilitating the already needed calculation of HH at the lower level of the hierarchy.
