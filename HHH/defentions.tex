\section{Notations and Problem Definition}

A  \textit{flow} is a set of packets that share a common property (usually IP header fields).  For example, all packets that have the same IP\_SRC address belong to the same flow. One can form a hierarchy of flows by considering  the prefixes  of the field that specify a flow. In our example, the prefixes of the IP\_SRC addresses forms the hierarchy.  More formally,  the   universe of flows, $\mathcal U$, forms a hierarchy where the \textit{dimensions} of the hierarchy is the number of fields that defines the flow and the \textit{hierarchy levels} are the various prefixes of the fields values.  Using this definition the flows in our example are of a single dimension (we only consider one field in the header) and a flow at level $l$ consists of all packets that share the same prefix of length $l$ from the IP\_SRC field.

In order to formally define the hierarchy, we define the \textit{Generalization Relation} between two hierarchy elements $p,q$, denoted by $p \prec q$, and say that $p$ generalizes $q$, if $p$ is a proper prefix of $q$ with respect to a specific dimension (i.e., a field in the packet header).  The \textit{hierarchy depth}, $H$, is the length of the maximal path of prefixes that confirms with this generalization relation. In our example, the depth of the hierarchy is $32$ as there are $32$ bits in an IP address. 

Given a \textit{stream} of packets, $\mathcal S$ (sometimes referred to also as a trace) one can ask what is the frequency of a given flow in that stream. That is, how many packets that belong to this specific flow appear in the stream. In the case of streams of packets we also consider the weighted version of the frequency of a flow, which is the total amount of bytes carried by packets that belong to this flow.  We use $T$ to donate the total frequencies of all elements, $N$ to denote the overall number of packets in the stream, and $B$ to denote the total amount of bytes in the stream. Usually $T=N$, however when we consider the weighted frequencies of flows $T=B$.

Given a threshold $\phi$, an element of the hierarchy (either a prefix or a specific flow) $x$, is a \textit{Heavy Hitter (HH)} if its frequency is at least $\phi$ of the total frequencies of all elements, i.e., $f_x \geq \phi T$. The \textit{Conditioned frequency} of prefix $p$ with respect to a set of prefixes $P$, $cf_p(P)$, is the sum of frequencies of all elements that $p$ generalizes without those of $P$. That is, we subtract from the frequency of $p$ the sum of frequencies of the flows that are generalized by elements of the set $P$.

Given a threshold $\phi$, an element of the hierarchy (either a prefix or a specific flow) is a \textit{Hierarchical Heavy Hitter (HHH)} if its conditioned frequency with respect to the set of HHH  is at least $\phi T$.  Another way to define the set of HHH for a given a stream of packets and a threshold $\phi$, is to build it recursively. The set $HHH_H$ is the set of HH at the lowest level of the hierarchy, i.e., $HHH_H=\{x\in \mathcal U| f_x \geq \phi T\}$. The set $HHH_l$, is the set of all prefixes at level $l$ that their conditional frequency with respect to $HHH_{l+1}$ is at least $\phi T$, i.e., $HHH_l=\{p \text{ at level } l | cf_p(HHH_{l+1}) \geq \phi T\}$. Then, the set of all HHH from all the levels of the hierarchy is $HHH=\bigcup_{l=0}^{H}HHH_l$.
 Table~\ref{tab:def} contains the list of symbols and notations we use throughout this chapter.
\begin{table}[ht]
	%\footnotesize
	\centering{
	\resizebox{1.0 \columnwidth}{!}{
	\begin{tabular}{|c|l|}
		\hline
		Symbol & Meaning \tabularnewline
		\hline
		$\mathcal U$ & The universe of flow identifiers \tabularnewline
		\hline
		$\mathcal S$ & The stream of packets \tabularnewline
		\hline
		$T$ & The total frequencies in $\mathcal S$\tabularnewline
		\hline		
		$N$ & The overall number of packets in $\mathcal S$\tabularnewline
		\hline		
		$B$ & The total byte count of packets in $\mathcal S$\tabularnewline
		\hline
		$f_x$ & The frequency of flow $x\in\mathcal U$ \tabularnewline
		\hline
		$cf_p(P)$ & The conditional frequency of $p$ in respect to $P$ \tabularnewline
		\hline
		$\phi$ & Heavy Hitter threshold  \tabularnewline
		\hline	
	    $H$ & The depth of the hierarchy \tabularnewline
		\hline
		$p \prec q$ & $p$ generalize $q$ in the hierarchy \tabularnewline
		\hline
		$C$ & The number of available counters \tabularnewline
		\hline
	\end{tabular}
	}	
}
	\caption{A list of symbols and notations}
	\label{tab:def}
\end{table}
%\normalsize 
