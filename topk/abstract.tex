\section{abstract}

Monitoring network traffic is an important building block for various management and security systems. In typical settings, the number of active flows in a network node is much larger than the number of available monitoring resources and there is no practical way to maintain “per-flow” state at the node. This situation gave rise to the recent interest in streaming algorithms where complex data structures are used to perform monitoring tasks like identifying the top-$k$ flows using a constant amount of memory.  However, these solutions require complicated “per-packet” operations, which are not feasible in current hardware or software network nodes.

In this paper, we take a different approach to this problem and study the ability to perform monitoring tasks using efficient built-in counters available in current network devices. We show that by applying non-trivial control algorithms that change the filter assignments of these built-in counters at a fixed time interval, regardless of packet arrival rate, we can get accurate monitoring information. We provide an analytical study of the top-$k$ flows problem and show, using extensive emulation over recent real traffic, that our algorithm can perform at least as well as the best-known streaming algorithms without using complex data structure or performing expensive “per-packet” operations.